
%The next line specifies the style of document to produce and asks for 12pt type
\documentclass[12pt]{article}   


%This specifies a larger left margin to permit grader remarks to be added.
\usepackage[letterpaper, margin=1in]{geometry}

\author{Jeff Putlock}
\date{\today}
\title{\vspace*{-3cm}Open Source Analysis}

\usepackage{array}
\newcolumntype{L}{>{\centering\arraybackslash}m{10cm}}

\begin{document}
	
	\maketitle
	
	\section{Project Analysis}
	
	\begin{enumerate}
		\item Submitty
		
		\hspace*{-1cm}
		\noindent\begin{tabular}{ | c | c | L | } 
			\hline
			Evaluation Factor & Level (0-2) & Evaluation Data \\\hline
			Licensing & 2 & Submitty uses the BSD 3-clause license. \\\hline
			Language & 1 & Submitty uses mostly PHP and HTML with some Python / C++, which I would prefer to do. \\\hline
			Level of Activity & 2 & For the past year, every week has had a commit. \\\hline		
			Number of Contributors & 2 & There have been 93 contributors over the course of Submitty's development. \\\hline
			Product Size & 2 & This project has many lines of code and the total repository amounts to 38 megabytes of data. \\\hline
			Issue Tracker & 2 & There are 278 open issues and 1427 closed. There have been multiple issues opened and closed within the past few days. \\\hline
			New Contributor & 2 & There are multiple links around that point you to "suggestions for new developers." \\\hline
			Community Norms & 1 & There are no blatant examples of poor behavior but also no stated code of conduct. \\\hline
			User Base & 2 & Submitty has the RPI Computer Science student body as its userbase and is growing to other sections of the school and to other universities. The slack channel has open communication between people struggling to use it or set it up, and there are detailed instruction lists. \\\hline
			Total Score & 16 / 18& \\\hline
		\end{tabular}
	
	\newpage
	
		\item YACS - Yet Another Course Scheduler
		
		\hspace*{-1cm}
		\noindent\begin{tabular}{ | c | c | L | } 
			\hline
			Evaluation Factor & Level (0-2) & Evaluation Data \\\hline
			Licensing & 2& YACS uses the AGPL license. \\\hline
			Language & 0 & It is written mostly in Ruby and also using Typescript, neither of which I am particularly interested or skilled in. \\\hline
			Level of Activity & 1 & It was active for a large portion of the year except the past few weeks. \\\hline		
			Number of Contributors & 2 & There have been 39 contributors over the course of development. \\\hline
			Product Size & 2 & The project is fairly large, amounting to approximately 20 megabytes of data. \\\hline
			Issue Tracker & 1 & There are some issues being added and removed but fairly slowly. \\\hline
			New Contributor & 2 & The project clearly outlines where new contributors should go to get started and specifically mark off which issues are beginner-friendly. \\\hline
			Community Norms & 2 & The project clearly displays a Code of Conduct and has no obvious poor behavior. \\\hline
			User Base & 2 & The entirety of the RPI community uses YACS as their course scheduler, and NYU will soon be using it as well. \\\hline
			Total Score & 14 / 18 & \\\hline
		\end{tabular}
	
		\item TensorFlow
		
		\hspace*{-1cm}
		\noindent\begin{tabular}{ | c | c | L | } 
			\hline
			Evaluation Factor & Level (0-2) & Evaluation Data \\\hline
			Licensing & 2 & TensorFlow is licensed under the Apache 2.0 license. \\\hline
			Language & 2 & It is written mainly in both C++ and Python. \\\hline
			Level of Activity & 2 & It is constantly being updated with well over one hundred commits each week. \\\hline		
			Number of Contributors & 2 & There are 2043 contributors working on the project. \\\hline
			Product Size & 2 & The project makes up a whopping 350 MB. \\\hline
			Issue Tracker & 2 & There are plenty of issues being reported constantly, and many being closed each day. \\\hline
			New Contributor & 2 & Inside of the contributing guidelines page, there are links that send you to beginner friendly issues to resolve. \\\hline
			Community Norms & 2 & There is a clear Code of Conduct linked multiple times, with community standards also posted afterwards. \\\hline
			User Base & 2 & Many companies and users use TensorFlow across the world to create neural networks. \\\hline
			Total Score & 18 / 18 & \\\hline
		\end{tabular}
	
	\end{enumerate}
	
	\section{In-Depth Analysis}
	
\end{document}